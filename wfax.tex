% 
%jff-notes
%
\documentclass[11pt]{article}
\usepackage[pdftex]{graphicx}
\usepackage{amssymb}
\usepackage{latexsym}
%\usepackage{relsize}
\usepackage{textcomp}
%processed for 10 pt 
%\documentstyle[epsf,psfig]{article}
%\documentstyle[epsf]{article}
\oddsidemargin 0pt
\topmargin -0.0cm
\textwidth 6.2in
\textheight 8.5in
\baselineskip 18pt
%\renewcommand{\baselinestretch} {1.5}
\newenvironment{nitemize}
   {\begin{list}{\begin{math}\bullet\end{math}}%
      {\setlength{\leftmargin}{5mm}
       \setlength{\topsep}{1mm}
       \setlength{\parsep}{0in}
       \setlength{\itemsep}{.7mm}}}%
   {\end{list}}

\newcommand{\fract}[2]{\frac{\textstyle #1}{\textstyle #2}}
\newcommand{\trans}[3]{#1 \stackrel{#2}{\longrightarrow} #3}
\newcommand{\notrans}[3]{#1 \stackrel{#2}{\not\! \longrightarrow} #3}
\bibliographystyle{plain}
\begin{document}
\title{A simple weatherfax plugin for SDRuno}
\author{
Jan van Katwijk\\
Lazy Chair Computing \\
The Netherlands\\
{\em J.vanKatwijk@gmail.com}}
%\date{}
\maketitle
%\baselineskip 22pt
\ \\
\ \\
\includegraphics[width=140mm]{wfax-example.png}
\ \\
\section{Introduction}
The SDRuno weatherfax plugin is a simple plugin to decode weatherfax signals.

On shortwave (between 3 and 16 Mhz) there are
transmissions of weatherfax charts on a regular base.
Here, in western Europe, 3588 KHz and 4601 KHz are frequencies where
(almost) continuously weather charts are transmitted and can be received.
(Where I live people are extremely climate aware, so the number of
solar panels in overwhelming, and so is the amount of jamming signals,
from time to time making it hard to receive a chart noise free).

The most common format for transmitting weather charts is
{\em Wefax576}, a format with an IOC of 576 and 1200 lines charts.
The so-called IOC
(index of cooperation) leads to a width if the chart of
app 1800 pixels.

Transmission is with 2 lines a minute, and a chart has 1200 lines
so, with some header information preceding the chart, transmission
takes more than 10 minutes.

Modulation of the signal is by phase shifting, with a signal deviation of
app  +400 Hz for white, and -400 Hz for black.

A transmission starts with a signal, a few seconds, for Wefax576
precise 300 Hz,
followed by a number of phase lines with which the receiver can synchronize
with the transmission.
Such a {\em phase line} starts and ends with 2.5 \% of the linelength
with a pure white signal, and 95 \% of the linelength with pure black.

The end of the transmission, for Wefax576, is with a tone of 450 Hz.

\section{SDRuno setting}
\subsection{Setting the samplerate}
The implementation uses a sample frequency of {\em 12 KHz},
similar to e.g. the navtex and the rtty plugin.
The SDRuno software can provide a samplerate
of 62.5 KHz, remaining filtering and decimation is done in
the weatherfax plugin itself.
The SDRuno setting is then to a samplerate of 2 MHz, and a decimation factor
32.

\includegraphics[width=100mm]{main-widget.png}

The plugin generates an audiotone of 800 Hz + the tuning offset, in my
experience is sound very helpful in precise tuning.

The sound is output with a rate of 48000, setting "AM" in the RX control window
will set this rate.

\subsection{Setting the frequency}

As said, the weatherfax frequencies are predefined, so just select
a frequency from one of the lists of frequencies.
In 
\begin{verbatim}
https://www.weather.gov/media/marine/rfax.pdf
\end{verbatim}

Note that - differing from some other implementations - tuning is
precisely on the mentioned frequency, not 1900 Hz less!
\section{The plugin}
The plugin widget is shown in the picture below

\includegraphics[width=140mm]{wfax-example-2.png}

The widget for this plugin is large, the weather chart is displayed
on it. The size of the widget is such that a wefax576 chart is shown
on precisely one quarter of its size (app 900 pixels wide, 600 lines).

The top lines of the widget are reserved for the controls,
the top line contains:
\begin{itemize}
\item a selector for the kind of charts, default Wefax576, alternatively
(but untested) wefax288;
\item the {\em modulation mode}, default FM, alternatively AM;
\item the {\em phase}, default is the higher frequency of the signal
used for white, and the lower for black. Selecting "invers" reverses this
interpretation;
\item {\em BW}, a choice between Black and white, gray and color;
]item the {deviation}. While in Europe the deviation of the modulation is
400 Hz, literature states that the US uses 450 Hz.
\item a blank field (for later use);
\item a label displaying the {\em state} of the decoder. 
\begin{itemize}
\item {\em APTSTART} is - as the name suggests - the start state. The software
will read incoming signals until a few seconds a signal of 300 Hz
is received;
\item {\em PHASING} is the state where the software is trying to synchronize.
If - during a longer time - no reliable synchronization can be realized, the
asumtpion is that the detection of the 300 Hz was erroneous, and the APTSTART
state is entered;
\item {\em ON SYNC} is the state when there is a reliable synchronization, and
in this state the data lines are processed.
The lines in the picture will be displayed on the widget.
\item {\em FAX\_DONE} is - as the name suggests - the stated entered when
processing the picture finishes. If {\em saving} was set, the picture
will be stored in a file and the APTSTART state will be entered again.
If {\em saving} was not set, the software will wait in the final state
until a {\em reset} is given.
\end{itemize}
\end{itemize}
The second line contains 5 elements
\begin{itemize}
\item the {\em save} button. When set, the software will continuously
run the sequence to decode a picture and store each picture in a file.
If set the text on the button reads {\em saving}, otherwise {\em Save}.
\item the {\em reset} button, which does what
can be expected from a reset button;
\item a label on which - while in the {\em APTSTART} state - the frequency
of the incoming decoded signal is displayed.
\item a label on which - when in sync - the line number
of the line currently being decoded is displayed;
\item a {\em cheat} button. As said, processing a whole chart in mode wefax576
takes well over 10 minutes. The cheat button cheats the system by forcing
it into state {\em ON SYNC}.
\end{itemize}
\end{document}


